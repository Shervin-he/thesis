\cchapter{مقدمه}
\par
با معرفی بیت‌کوین به عنوان یک ارز دیجیتال بدون پشتوانه و ارزش ذاتی در سال ۲۰۰۸ و فراگیر شدن استفاده‌ی از این بستر برای تراکنش‌های مالی مطالعات بسیاری در مورد دلیل موفقیت آن شد. اما با گذشت زمان توجه‌ها بیشتر به تکنولوژی مورد استفاده‌ی این ارز دیجیتال و به طور خاص بلاک‌چین جلب شد.
از استفاده‌های دربلاک‌چین بیت کوین برای تولید ابزارهای مالی جدید می‌توان به سکه‌های رنگی به عنوان ارزهای جدید و
 \lr{Namecoin}
  برای بستر خرید و فروش دامنه‌ی وب‌سایت اشاره کرد. 
\par
استفاده از بلاک‌چین به عنوان یک لیست تغییرناپذیر به کمک اثبات کار یک راه‌حل توزیع‌شده برای مسئله‌ی ژنرال‌های بیزنتین را ایجاد کرد که خود باعث تولید ارزهای جدید به روی بسترهای مستقل شده و برای کاربردهای جدید شد. یکی از بلندپروازانه‌ترین ایده‌هایی که تا به امروز دیده شده اتریوم است. تراکنش‌های بیت‌کوین توانایی ثبت اسکرپت‌هایی که قواعدی برای تراکنش ثبت کنند را دارند ولی تعدادی از خصوصیت‌های معمول زبان‌های برنامه‌نویسی
\lr{turing-complete}
 مانند حلقه را پشتیبانی نمی‌کنند. هدف از ساخت اتریوم ساخت یک زبان برنامه نویسی
 \lr{ turing-complete }
 برای این بستر است. 
\par
فلسفه‌ی ساخت پروتکل اتریوم رو می‌توان در این ۵ پایه خلاصه کرد: 
\begin{itemize}
	\item
سادگی: پروتکل باید برای برنامه‌نویسان ساده و دردسترس باشد حتی به قیمت از کم شدن بهره‌وری کل سیستم.
\item  
کامل بودن: اتریوم باید یک زبان
\lr{turing-complete}
  داشته باشد و هر مدل ریاضی را بتوان با آن پیاده کرد. 
\item
بخش‌پذیری: قسمت‌های اتریوم باید از هم جدا باشند و توانایی عوض کردن الگوریتم‌های و ساختارداده‌های سیستم مانند درخت پاتریشا وجود داشته باشد، بدون این که قسمت‌های دیگر سیستم از این تغییر باخبر شوند
\item
چابکی: جزییات پروتکل اتریوم باید قابل تغییر باشند. 
\item
برابری: سیستم نباید فعلانه جلوی یک دسته از کاربردها رو بگیرد یا آن‌ها رو محدود کند.
\end{itemize}
\par
با بوجود آمدن اتریوم به عنوان یک بستر کامل، بی‌اعتماد و توزیع‌شده برای قرارداد‌های هوشمند کاربرد‌های اشاره شده در بالا را می‌توانن به سادگی با نوشتن چند خط کد پیاده کرد. این سادگی در پیاده‌سازی باعث جذب بسیاری از توسعه‌دهندگان می‌شود که می‌توانند کاربردهای جدیدی پیاده‌ کنند که به عنوان یک کارپرداز خودکار در این بستر فعالیت کنند. تغییرناپذیری قرارداد‌هایی که در بستر بلاک‌چین نوشته می‌شوند باعث اعتماد مشتریان آن‌ها به آن قرارداد می‌شود ولی این تغییرناپذیری به معنی این است که اگر قرارداد «اشتباهی» در این بستر نوشته شود راهی برای تصحیح آن نیست. برای مثال در سال ۲۰۱۶ به اندازه‌ی ۵۰ میلیون دلار اتر از یک سازمان کرودفاندینگ در اثر یک باگ امنیتی از بک قرارداد آن‌ها دزدیده شد. با توجه به تغییر ناپذیر بودن بلاک‌چین هیچ راهی جز تغییر پروتکل برای بازگرداندن پول وجود نداشت و در نهایت با یک انشعاب سخت از این بستر پول به آن مجموعه بازگردانده شد. این تصمیم برای تغییر سیستم باعث شد کاربران اتریوم به دو دسته تقسیم شوند، دسته‌ی اول کسانی که از بازگردانده شدن پول به سازمان حمایت می‌کردند و بلاک‌چین جدید رو به عنوان بلاک‌چین اصلی اتریوم قبول کردند و دسته‌ی دوم که با این استدلال که قانون اتریوم کد قراردادهاست و چون قرارداد به درستی اجرا شده باید آن مبلغ به هکرها تعلق بگیرد، بلاک‌چین جدید را قبول نکرده و بلاک‌چین قبلی را ادامه دادند. از نمونه‌های دیگر این مسئله می‌توان به قفل شدن ۳۰۰ میلیون دلار اتر متعلق به شرکت
\lr{parity}
 در نوابر ۲۰۱۷ اشاره کرد. 
 \\
 لازم به ذکر است که هیچ کدام از مشکلات امنیتی نام برده شده مشکل خود بستر اتریوم نبوده و مسئله استفاده‌ی اشتباه از زبان برنامه‌نویسی آن و قابلیت‌های آن‌ها بوده است. با این وجود توجه به مسائل امنیتی در این بستر ناآشنا و جدید با توجه به طرز فکر متفاوت از برنامه‌نویسی عادی بسیار مهم است. 
 \\
 در ادامه‌ی این تحقیق به بررسی دقیق‌تر بعضی از این مشکلات امنیتی خواهیم پرداخت. 
 \par
 
 یک سوال مهم در زمینه‌ی قراردادهای هوشمند اتریوم کاربردهای ممکن و یا مناسب این بستر است. از کاربرداهای معروف این بستر به کاربرد‌های زیر می‌توان اشاره کرد: 
 \begin{itemize}
 	\item 
 	ارز‌های جدید
 	\item
 	سیستم‌های هویت
 	\item
 	فایل‌سیستم‌های توزیعشده
 	\item
 	سازمان‌های خودکار توزیع‌شده
 \end{itemize}
\par
در ادامه‌ی این تحقیق ابتدا به تعریف مفاهیم پرکاربرد آن می‌پردازیم و در ادامه ... 

\cchapter{تعریف مفاهیم}
\par
در این بخش تعریف مفاهیم مورد استفاده‌ در این تحقیق می‌پردازیم. 
\begin{itemize}
	\item 
	\textbf{بلاک‌چین}:
	بلاک‌چین یک ساختارداده متشکل‌ از بلوک‌های پشت‌‌سرهم که هر بلوک شامل هشی از خودش بلوک قبلی هم هست. در نتیجه به تغییر یک بلوک باید تمام بلوک‌‌های بعد از آن را تغییر داد تا ساختار درست باشد. 
	\item
	\textbf{اثبات کار}:
	روش اثبات کار بر اساس
	\lr{hashcash}
	 که یک روش برای جلوگیری از حملات
	  \lr{DDoS}
	   طراحی شده بود ساخته شده است. روش کار 
	   \lr{hashcash}
	   به شکل زیر است:
	   \\
	   برای این که یک ایمیل توسط سرور ارسال شود همراه متن ایمیل کلاینت باید که رشته‌ای ارسال کند که اگر  هش 
	   \lr{SHA-1}
	   آن از آن گرفته شود ۲۰ بیت اول آن صفر خواهند بود. به دلیلی تصادفی بودن هش رشته باید با امتحان کردن رشته‌‌های مختلف به یک رشته‌ی مناسب برسد. زمان حل این مسئله‌ برای کامپوتر‌های 
	   \lr{1 GHZ}
	   آن زمان حدود یک ثانیه بود و زمان بررسی درست بودن آن هش تنها ۲ میکروثانیه است.
	   \\
	   برای یک کاربر عادی که قصد ارسال یک ایمیل را دارد زمان یک‌ ثانیه‌ای قابل‌قبول است اما اگر یک مهاجم قصد 
	   \lr{spam}
	   کردن توسط این سرویس را داشته باشد زمان یک ثانیه برای هر ایمیل هزینه‌ی بسیار بالایی خواهد بود.
	   \\
	   در بستر بیت‌کوین از این روش برای توافق بر بلوک‌های بعدی بلاک‌چین به صورت زیر استفاده می‌شود:
	   \\
	هر بلوک جدید حاوی تعدادی تراکنش برای ثبت در بلاک‌چین توسط ماینترها به یک بلوک تبدیل می‌شود. ولی برای این که این بلوک توسط بقیه پذیرفته شود باید در این بلاک یک nounce قراردهند به صورتی که هش بلاک از یک عددی که توسط پروتکل بیت‌کوین انتخاب می‌شود کمتر باشد. این شرط در طول زمان به صورت خودکار به روزرسانی می‌شود به طوری که در هر لحظه به صورت میانگین اضافه کردن بلاک ۱۰ دقیقه از کل شبکه زمان ببرد. از آنجایی که تنها راه پیدا کردن همچین رشته‌ای بروت‌فورس است، توان محاسباتی بالاتر باعث شانس بیشتر برای پیدا کردن بلاک بعدی خواهد شد. 
	
	
	\item 
	
\textbf{مسئله‌ی جنرال‌های بیزنتین}

	مسئله‌ی جنرال‌های بیزنتین یا تحمل خطای بیزنتین مدلی از تحمل خطا در سیستم‌های توزیع شده است. در این مسئله تعدادی جنرال یک ارتش با هم به صورت پیام‌های یک به یک صحبت می‌کنند و در ساده‌ترین حالت در مورد حمله کردن یا عقب‌نشینی در یک نبرد تصمیم می‌گیرند. ولی تعدادی از این جنرال‌ها خائن بوده و تلاش می‌کنند که جمع به توافق غلطی برسد (توافق درست توافقی است که اگر هیچ خائنی وجود نداشت به آن می‌رسیدند) و یا با جواب ندادن مانع تصمیم‌گیری آن‌ها شوند. در ساده‌ترین حالت و بدون استفاده از امضا‌های دیجیتال ثابت می‌شود که برای 3k + 1 جنرال، با رای‌گیری می‌توان تا k خائن را تحمل کرد. 
	راه‌حل خلاقانه‌ی بیت‌کوین برای حل این مسئله استفاده از بلاک‌چین برای ذخیره‌ی اطلاعات و استفاده از اثبات کار برای اضافه کردن بلوک به بلاک‌چین است. 
	\\
	برای نشان دادن نحوه‌ی حل این مسئله‌ یک مثال را بررسی می‌کنیم. فرض می‌کنیم شخص A یک بیت‌کوین را به B منتقل کرده و این تراکنش در بلاک‌چین ثبت شده و در ازای آن کالایی دریافت کرده، حال قصد دارد که این تراکنش رو از بلاک‌چین بیت‌کوین حذف کند تا بتواند آن را ۲ بار خرج کند. از آنجایی که نود‌های شبکه‌ی بیت‌ککوین اگر ۲ زنجیره از بلوک‌ها دریافت کنند زنجیره‌ی بلند‌تر را قبول خواهند کرد باید ۲ بلوک سالم بسازد قبل از این که کل شبکه یک بلوک به شبکه اضافه کنند. 
	\\
	احتمال موفقت حمله‌ی A مساوی
	$(\frac{A's\ computational\ power}{Bitcoin\ network's\ computational\ power}) ^ 2 $
	است. اگر توان محاسباتی A از بقیه‌ی شبکه کمتر باشد این کسر یک عدد کوچک‌تر از 0.5 است. اگر در این کار به موقع موفق نشود سه بلاک عقب می‌افتد و توان فرمول بالا تبدیل به سه می‌شود و احتمال موفقیتش کمتر از پیش نیز می‌شود. 
	\\
	این مسئله مسئله‌ی قمارباز نام دارد که نشان داده می‌شود در آن در طول زمان احتمال موفقت مهاجم به صورت نمایی کاهش پیدا می‌کند.
	
	\item 
	\textbf{انشعاب}:
	منظور از انشعاب در ارز‌های دیجیتال تبدیل یک بلاک‌چین به دو بلاک‌چین است، گاها برای ساخت ارز‌های جدید از بلاک‌چین موجود یک ارز دیگر مثل بیت‌کوین استفاده‌ می‌شود، این کار باعث می‌شود که شروع بلاک‌چین آسان‌تر و امن‌تر شود. در روال عادی کار بیت‌کوین نیز ممکن است انشعابی رخ دهد اما هر ماینری که متوجه انشعابی شود به صورت خود‌کار بلند‌ترین زنجیره را به عنوان زنجیره‌ی درست انتخاب می‌کند. در صورتی که یک انشعاب برای تولید بلاک‌چین جدید انجام گیرد و بلوک‌هایی قبلی که در آن وجود داشتند همچنان درست حساب شوند این انشعاب را انشعاب نرم و اگر بلوک‌های قبلی مورد قبول سیستم‌ جدید نباشند انشعاب را انشعاب سخت می‌نامیم.
	\item 
	\textbf{ماین‌کردن}: 
	به عملیات پیدا ساختن بلوک‌های جدید روی بلاک‌چین به هدف پیدا کردن بلاک‌های درست و دریافت جایزه‌ی آن‌ها ماین‌کردن می‌گوییم.
	\item 
	\textbf{ماینینگ‌ پول}:
	از آن‌ جایی که ماین کردن برای یک نفر با توجه به احتمال پایین این که بتوانند بلاک معتبر را زودتر از بقیه‌ی شبکه پیدا کنند بسیار پایین است، ماینینگ‌پول‌ها شکل گرفته‌اند. با تقسیم کردن کار بین چندین ماشین شانس پیدا کردن بلوک معتبر بیشتر می‌شود و جایزه‌ی ماین‌ کردن به نسبت توان محاسباتی بین شرکت‌کنندگان تقسیم می‌شود. برای بدست آوردن توان محاسباتی که هر ماشین برای این کار مصرف کرده از تعداد بلوک‌هایی که هش آن‌ها به اندازه‌ی کافی برای درست بودن کوچک نیست ولی به جواب درست نزدیکند استفاده می‌شود.
	
	\item
	\textbf{قرارداد هوشمند}
	لفظ قرارداد‌های هوشمند اولین بار در سال ۱۹۹۳ توسط N.Szabo به عنوان یک پروتکل تراکنش کامپیوتری که شروط یک قرارداد را اجرا می‌کند. در اولین مثال معروف قرارداد‌های هوشمند یک وندینگ‌ ماشین را مثال زد که در ازا ی سکه‌ی به طور اتوماتیک کالای مورد نظر را به مشتری می‌دهد، همینطور از آنجایی که بدون پول دادن هرگز کالایی نمی‌دهد و امنیت سکه‌ها را از طریق صندوق خود تا حد معقولی تامین می‌کند قرارداد مناسبی بین مشتری و تولیدکننده‌ی کالا محسوب می‌شود.
	هدف نهایی قراردادهای هوشمند کاهش نیاز به اعتماد کردن و افراد میانی در یک قرارداد است و با بوجود آمدن بسترهای ارز دیجیتال و راه‌حل‌های جدید مسئله‌ی جنرال‌های بیزنتین بستر مناسبی برای ساخت قراردادهای هوشمند و توزیع‌شده بدون نیاز به اعتماد به شخص ثالث بوجود آمده است. 
	با وجودی که به کمک زبان اسکریپتینگ بیت‌کوین می‌توان مدل‌های مختلفی از قرارداد‌های هوشمند را تولید کرد، با اتریوم به به کمک زبان برنامه‌نویسی turing-complete آن در تئوری می‌توان هر قرارداد هوشمند ممکن را تولید کرد. 
	
	
\end{itemize}

\cchapter{معرفی قابلیت‌های قالب}
در این بخش با آوردن یک متن ساده به نمایش ظاهر و ساختار قالب و همچنین معرفی برخی دستورات لازم برای کار با آن قالب پرداخته می‌شود. این توضیحات بسیار مختصر بوده و صرفاً برای معرفی قالب می‌باشد و چنانچه با \lr{LaTeX} آشنایی ندارید، بهتر است پیش از آغاز تدوین پایان‌نامه مختصری در مورد نحوه کار با \lr{LaTeX} مطالعه بفرمایید.
\section{نگارش}
رعایت تمامی اصول نگارش در هنگام تدوین پایان‌نامه الزامیست، بسیاری از نکات نگارشی توسط قالب رعایت می‌شوند. در ادامه این بخش به معرفی برخی دستورات کاربردی برای این کار پرداخته می‌شود.
\\
به طور پیش‌‌فرض هر پاراگراف به صورت خودکار با فاصله  از کنار آغاز می‌شود.  چنانچه در حالت خاصی، نیاز به حذف این فاصله باشد می‌توانید از دستور 
\LTR{\lr{\textbackslash noindent}} \RTL{}\noindent
استفاده کنید.\\
برای نوشتن متون انگلیسی لازم است آن‌ها را داخل تگ
 \lr{\textbackslash lr\{\}}
 قرار دهید. به عنوان مثال با نوشتن 
 \lr{\textbackslash lr\{word\}}
 کلمه \lr{word} به درستی داخل متن قرار می‌گیرد.
 \par
 با نوشتن \textbackslash\textbackslash می‌توانید خط جدیدی را آغاز کنید و نوشتن \textbackslash par نیز باعث ایجاد یک پاراگراف جدید خواهد شد. با قرار دادن متن در داخل تگ \lr{\textbackslash textbf\{\}} متن به صورت \textbf{ضخیم} و با قرار دادن نوشته در داخل تگ \lr{\textbackslash textit\{\}} نوشته \textit{کج} خواهد شد. امکان استفاده هم‌زمان از این تگ‌ها نیز وجود دارد. برای آشنایی بیشتر با دستورات، پیشنهاد می‌شود به آموزش‌های \lr{LaTeX} مراجعه کنید.
 \par

 پانوشت‌‌ها یکی از بخش‌های اصلی در هر نوشته‌ای می‌باشد. در این قالب شماره‌های پانوشت در هر صفحه، مجدداً از ۱ آغاز می‌شود.  با نوشتن
\LTR{\lr{\textbackslash LTRfootnote\{footnote\}}} \RTL{} \noindent
  می‌توان یک پانوشت لاتین\LTRfootnote{footnote} اضافه نمود.
 \par
برای افزودن پانوشت‌های فارسی\RTLfootnote{یک پانوشت فارسی} نیز از دستور زیر استفاده می‌شود.
\LTR{\lr{\textbackslash RTLfootnote\{پانوشت\}}} \RTL{} \noindent
نمونه‌های افزودن پانوشت نیز در همین قسمت وجود دارد.






\section{بخش‌بندی}
برای ساخت یک فصل جدید کافیست از دستور
\LTR{\lr{\textbackslash cchapter\{عنوان\}}} \RTL{} \noindent
استفاده شود. با نوشتن این دستور به صورت خودکار یک فصل جدید اضافه شده و عنوان آن در یک صفحه مجزا قرار می‌گیرد.
هر بخش می‌تواند شامل تعدادی \lr{Section} باشد. شماره‌های آن مانند آنچه در بالا نیز مشاهده می‌کنید با . از یک‌دیگر جدا شده و به صورت خودکار شماره‌گزاری  شده و به فهرست اضافه می‌شوند. کافیست برای ساخت بخش دستور 
\LTR{\lr{\textbackslash section\{عنوان\}}} \RTL{}\noindent
را وارد کنید. همچنین با دستورات زیر می‌توانید زیربخش و حتی زیر زیر بخش، ایجاد کنید.
\LTR{\lr{\textbackslash subsection\{عنوان\}}} \RTL{}
\LTR{\lr{\textbackslash subsubsection\{عنوان\}}} \RTL{}\noindent
به عنوان مثال بخش زیر را در نظر بگیرید :(متون این زیربخش‌ها بی‌معنا و برای پر کردن صفحه می‌باشد.)

\subsection{یک زیر بخش}
اگر زیر بخش‌ها به سطح سوم برسند شماره‌گذاری نمی‌شوند ولی در فهرست مطالب قرار می‌گیرند، به عنوان مثال:
\subsubsection{زیر بخشی در زیر بخش}
متن
\subsubsection{ دومین زیربخش در زیربخش}
متنی دیگر

\subsection{زیر بخشی دیگر}
لورم ایپسوم ( به انگلیسی \lr{lorem ipsum} ) متنی بی مفهوم است که تشکیل شده از کلمات معنی دار یا بی معنی کنار هم. کاربر با دیدن متن لورم ایپسوم تصور میکند متنی که در صفحه مشاهده میکند این متن واقعی و مربوط به توضیحات صفحه مورد نظر است واقعی است. حالا سوال اینجاست که این متن « لورم ایپسوم » به چه دردی میخورد و اساسا برای چه منظور و هدفی ساخته شده است؟ پیش از بوجود آمدن لورم ایپسوم ، طراحان وب سایت در پروژه های وب سایت و طراحان کرافیک در پروژه های طراحی کاتولوگ ، بروشور ، پوستر و ... همواره با این مشکل مواجه بودند که صفحات پروژه خود را پیش از آنکه متن اصلی توسط کارفرما ارائه گردد و در صفحه مورد نظر قرار گیرد چگونه پر کنند؟؟ اکثر طراحان با نوشتن یک جمله مانند «این یک متن نمونه است» ویا «توضیحات در این بخش قرار خواهند گرفت» و کپی آن به تعداد زیاد یک یا چند پاراگراف متن میساختند که تمامی متن ها و کلمات ، جملات و پاراگراف ها تکراری بود و از این رو منظره خوبی برای بیننده نداشت و ضمنا به هیچ وجه واقعی به نظر نمیرسید تا بتواند شکل و شمایل تمام شده پروژه را نشان دهد. از این رو متنی ساخته شد که با دو کلمه ( به فارسی : لورم ایپسوم ) آغاز میشد وبا همین نام در بین طراحان وب و گرافیک شناخته و به سرعت محبوب شد. وب سایت های سازنده لورم ایپسوم میتوانند هر تعداد کلمه و پاراگراف که بخواهید به صوورت تکراری یا غیر تکراری برایتان بسازند و تحویلتان بدهند تا از آنها در پروژه هایتان استفاده کنید. ( لورم ایپسوم فارسی) متن های لورم ایپسوم را به زبان فارسی و علاوه بر زبان فارسی به انگلیسی ، عربی ، ترکی استانبولی و ... برایتان میسازد. زبان های دیگر نیز رفته رفته به بانک اطلاعاتی لورم ایسپوم فارسی اضافه خواهند شد.  لورم ایپسوم ( به انگلیسی \lr{lorem ipsum} ) متنی بی مفهوم است که تشکیل شده از کلمات معنی دار یا بی معنی کنار هم. کاربر با دیدن متن لورم ایپسوم تصور میکند متنی که در صفحه مشاهده میکند این متن واقعی و مربوط به توضیحات صفحه مورد نظر است واقعی است. حالا سوال اینجاست که این متن « لورم ایپسوم » به چه دردی میخورد و اساسا برای چه منظور و هدفی ساخته شده است؟ پیش از بوجود آمدن لورم ایپسوم ، طراحان وب سایت در پروژه های وب سایت و طراحان کرافیک در پروژه های طراحی کاتولوگ ، بروشور ، پوستر و ... همواره با این مشکل مواجه بودند که صفحات پروژه خود را پیش از آنکه متن اصلی توسط کارفرما ارائه گردد و در صفحه مورد نظر قرار گیرد چگونه پر کنند؟؟ اکثر طراحان با نوشتن یک جمله مانند «این یک متن نمونه است» ویا «توضیحات در این بخش قرار خواهند گرفت» و کپی آن به تعداد زیاد یک یا چند پاراگراف متن میساختند که تمامی متن ها و کلمات ، جملات و پاراگراف ها تکراری بود و از این رو منظره خوبی برای بیننده نداشت و ضمنا به هیچ وجه واقعی به نظر نمیرسید تا بتواند شکل و شمایل تمام شده پروژه را نشان دهد. از این رو متنی ساخته شد که با دو کلمه ( به فارسی : لورم ایپسوم ) آغاز میشد وبا همین نام در بین طراحان وب و گرافیک شناخته و به سرعت محبوب شد. وب سایت های سازنده لورم ایپسوم میتوانند هر تعداد کلمه و پاراگراف که بخواهید به صوورت تکراری یا غیر تکراری برایتان بسازند و تحویلتان بدهند تا از آنها در پروژه هایتان استفاده کنید. ( لورم ایپسوم فارسی) متن های لورم ایپسوم را به زبان فارسی و علاوه بر زبان فارسی به انگلیسی ، عربی ، ترکی استانبولی و ... برایتان میسازد. زبان های دیگر نیز رفته رفته به بانک اطلاعاتی لورم ایسپوم فارسی اضافه خواهند شد.  

\section{ارجاعات}
دادن ارجاعات با افزودن \lr{BibTeX} مرجع به فایل \lr{thesis.bib} و سپس آوردن نام آن انجام می‌شود. مانند همین ارجاعی که در این صفحه وجود دارد. ارجاعات می‌تواند به منابع انگلیسی\cite{expret} یا فارسی\cite{Vahedi87} باشد.

\section{نمایه‌ها}
در این بخش به نحوه افزودن فایل‌های خارجی و جداول به متن پرداخته می‌شود.
\subsection{فرمول‌نویسی}
نگارش فرمول‌ها در \lr{LaTeX} به صورتی که مشاهده‌ می‌کنید انجام می‌شود، در ادامه چند فرمول به عنوان نمونه نوشته می‌شود. تمامی قواعد و قوانین فرمول‌نویسی در \lr{LaTeX} بدون نیاز به هیچ‌گونه تغییری در این قالب قابل استفاده می‌باشد.
\begin{equation}
MAP(Q) = \frac{1}{|Q|} \sum_{j = 1}^{|Q|}
{
	\frac{1}{m_j} \sum_{k = 1}^{m_j} {Precision(R_jk)}
}
\label{eq:map}
\end{equation}
\begin{equation}
MI(sa,w) = \sum_{ A_{sa} = 0,1}{
	\sum_{A_{w} = 0,1}{
		p(A_{sa} , A_{w}) \log{\frac{p(A_{sa} , A_{w})}{p(A_{sa}) p(A_{w})}}
	}%inner sum
}%First sum
\label{eq:mi}
\end{equation}
برای ارجاع به فرمول‌ها می‌توان از دستور \lr{\textbackslash ref\{label\}} استفاده کرد. به عنوان مثال: معادله \ref{eq:map} از معادلات مهم است.
\subsection{تصاویر}
برای افزودن تصاویر می‌توانید از فایل‌‌های برداری مانند \lr{PDF} و یا تصاویر پیکسلی مانند \lr{PNG} و \lr{JPG} استفاده کنید. افزودن این تصاویر مانند تصویر زیر انجام می‌شود و به راحتی می‌توان برای آنها توضیح نوشت و به طور خودکار در فهرست تصاویر قرار می‌گیرند. این تصاویر به طور خودکار در مکان مناسب قرار می‌گیرند.
\par
آوردن نام تصاویر در متن نیز مانند فرمول‌ها می‌باشد به عنوان مثال: تصویر \ref{fig:logo} لوگو دانشگاه را نشان می‌دهد.
\begin{figure}[tb]
	\centering
	\includegraphics[width=.6\linewidth]{logo}
	\caption {یک نمونه تصویر}
	\label{fig:logo}
\end{figure}
\par
متنی که در ادامه این بخش آمده فقط برای پرکردن فضا می‌باشد: لورم ایپسوم ( به انگلیسی \lr{lorem ipsum} ) متنی بی مفهوم است که تشکیل شده از کلمات معنی دار یا بی معنی کنار هم. کاربر با دیدن متن لورم ایپسوم تصور میکند متنی که در صفحه مشاهده میکند این متن واقعی و مربوط به توضیحات صفحه مورد نظر است واقعی است. حالا سوال اینجاست که این متن « لورم ایپسوم » به چه دردی میخورد و اساسا برای چه منظور و هدفی ساخته شده است؟ پیش از بوجود آمدن لورم ایپسوم ، طراحان وب سایت در پروژه های وب سایت و طراحان کرافیک در پروژه های طراحی کاتولوگ ، بروشور ، پوستر و ... همواره با این مشکل مواجه بودند که صفحات پروژه خود را پیش از آنکه متن اصلی توسط کارفرما ارائه گردد و در صفحه مورد نظر قرار گیرد چگونه پر کنند؟؟ اکثر طراحان با نوشتن یک جمله مانند «این یک متن نمونه است» ویا «توضیحات در این بخش قرار خواهند گرفت» و کپی آن به تعداد زیاد یک یا چند پاراگراف متن میساختند که تمامی متن ها و کلمات ، جملات و پاراگراف ها تکراری بود و از این رو منظره خوبی برای بیننده نداشت و ضمنا به هیچ وجه واقعی به نظر نمیرسید تا بتواند شکل و شمایل تمام شده پروژه را نشان دهد. از این رو متنی ساخته شد که با دو کلمه ( به فارسی : لورم ایپسوم ) آغاز میشد وبا همین نام در بین طراحان وب و گرافیک شناخته و به سرعت محبوب شد. وب سایت های سازنده لورم ایپسوم میتوانند هر تعداد کلمه و پاراگراف که بخواهید به صوورت تکراری یا غیر تکراری برایتان بسازند و تحویلتان بدهند تا از آنها در پروژه هایتان استفاده کنید. ( لورم ایپسوم فارسی) متن های لورم ایپسوم را به زبان فارسی و علاوه بر زبان فارسی به انگلیسی ، عربی ، ترکی استانبولی و ... برایتان میسازد. زبان های دیگر نیز رفته رفته به بانک اطلاعاتی لورم ایسپوم فارسی اضافه خواهند شد.  لورم ایپسوم ( به انگلیسی \lr{lorem ipsum} ) متنی بی مفهوم است که تشکیل شده از کلمات معنی دار یا بی معنی کنار هم. کاربر با دیدن متن لورم ایپسوم تصور میکند متنی که در صفحه مشاهده میکند این متن واقعی و مربوط به توضیحات صفحه مورد نظر است واقعی است. حالا سوال اینجاست که این متن « لورم ایپسوم » به چه دردی میخورد و اساسا برای چه منظور و هدفی ساخته شده است؟ پیش از بوجود آمدن لورم ایپسوم ، طراحان وب سایت در پروژه های وب سایت و طراحان کرافیک در پروژه های طراحی کاتولوگ ، بروشور ، پوستر و ... همواره با این مشکل مواجه بودند که صفحات پروژه خود را پیش از آنکه متن اصلی توسط کارفرما ارائه گردد و در صفحه مورد نظر قرار گیرد چگونه پر کنند؟؟ اکثر طراحان با نوشتن یک جمله مانند «این یک متن نمونه است» ویا «توضیحات در این بخش قرار خواهند گرفت» و کپی آن به تعداد زیاد یک یا چند پاراگراف متن میساختند که تمامی متن ها و کلمات ، جملات و پاراگراف ها تکراری بود و از این رو منظره خوبی برای بیننده نداشت و ضمنا به هیچ وجه واقعی به نظر نمیرسید تا بتواند شکل و شمایل تمام شده پروژه را نشان دهد. از این رو متنی ساخته شد که با دو کلمه ( به فارسی : لورم ایپسوم ) آغاز میشد وبا همین نام در بین طراحان وب و گرافیک شناخته و به سرعت محبوب شد. وب سایت های سازنده لورم ایپسوم میتوانند هر تعداد کلمه و پاراگراف که بخواهید به صوورت تکراری یا غیر تکراری برایتان بسازند و تحویلتان بدهند تا از آنها در پروژه هایتان استفاده کنید. ( لورم ایپسوم فارسی) متن های لورم ایپسوم را به زبان فارسی و علاوه بر زبان فارسی به انگلیسی ، عربی ، ترکی استانبولی و ... برایتان میسازد. زبان های دیگر نیز رفته رفته به بانک اطلاعاتی لورم ایسپوم فارسی اضافه خواهند شد.  


\subsection{جداول}
ساخت جداول در این قالب مانند بقیه متون نوشته شده توسط \lr{LaTeX} می‌باشد. ابزارهای مختلفی نیز برای ساخت خودکار این جداول موجود می‌باشد که می‌توانید از آنها  برای ساخت جداول پیچیده‌تراستفاده کنید. در ادامه نمونه‌ای جدول آورده می‌شود. برای ارجاع به نام جداول در متن نیز مانند تصاویر عمل می‌شود. به عنوان مثال در ادامه جدول \ref{tab:tch} را مشاهده می‌کنید.

\begin{table}[h]
	\begin{center}
%		\def\arraystretch{2}
		\caption{نمونه‌ای جدول}
		\begin{tabular}{|c|c|c|c|c|}
			\hline
			روش & 
			دقت & صحت & حساسیت & معیار اِف \\
			\hline
			
			روش اول & 
			\% 98/70 & 0/8821 & 0/8817 & 0/8819 \\
			\hline
			
			روش دوم & 
			\% 98/15 & 0/7464 & 0/4772 & 0/6118 \\
			\hline
		\end{tabular}
		\label{tab:tch}
	\end{center}
\end{table}

