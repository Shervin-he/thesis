\cchapter{مقدمه}
\par
امنیت در رای‌گیری همواره یک مسئله‌ی پیچیده بوده است که نیازمند یک فرد قابل اعتماد برای برگزاری و یک پروتکل امن برای جلوگیری از تقلب یا اشتباه در فرایند آن است. سیستم‌های رای‌گیری الکترونیک 

از سال ۱۹۶۰ وجود داشتند و اولین استفاده‌ بزرگ از آن‌ها در چند ایالت آمریکا در سال ۱۹۶۴ برای انتخابات ریاست جمهوری بود. رای‌گیری الکترونیک به‌سادگی می‌تواند هزینه برگزاری انتخابات را از طریق سادگی شمارش کاهش دهد. 

\section{فرایند رای‌گیری ایده‌آل}

شروط فرایند رای‌گیری ایده‌آل عبارت‌ است از:
\begin{itemize}
	\item 
	هر فرد واجد شرایط دقیقا یک بار بتواند رای دهد.
	\item 
	هیچ کسی نتواند به جای فرد دیگری رای دهد.
	\item 
  	هیچ فردی مجبور به رای دادن نشود.
  	\item 
  	هیچ فردی مجبور به رای دادن به کاندیدای خاصی نشود.
  	\item 
  	از شمارش هر رای اطمینان حاصل شود.
  	\item 
    نتیجه‌ی آرا ناشناس  باقی بماند. 
  	\item 
  	بسته به نیاز بتوان نتایج لحظه‌ای انتخابات را (بدون آسیب به شرط‌های قبلی) دید.
\end{itemize}

\section{سیستم‌های رای‌گیری سنتی}
\par
در رای‌گیری غیر الکترونیکی معمولا فرایند به شکل زیر است:
\\
فرد برای رای‌دادن به یکی از حوزه‌های رای‌گیری مراجعه کرده و با ارائه‌ی مدارک شناسایی خود یک برگه‌ی رای دریافت می‌کند. برگه رای دارای دو بخش است: قسمتی که برای ردیابی با اطلاعات شخصی فرد پر می‌شود و یک قسمت بی‌نام که فرد کاندیدای مورد نظر خود را در آن ثبت کرده و در یک صندوق می‌اندازد. 
\\
با بررسی مدارک شناسایی، شرط دوم فرایند رای‌گیری ایده‌آل تایید شده و با ثبت شدن اطلاعات فرد به عنوان یک رای‌دهنده از رای دادن دوباره‌ی او جلوگیری می‌شود. امنیت شخصی افراد در حوزه توسط برگزارکننده‌ی انتخابات تامین می‌شود و با وجود گزینه‌ی «رای سفید» فردی مجبور به رای دادن و یا رای دادن به یک کاندیدای خاص نمی‌شود. 
\\
با وجود یک صندوق برای چندین رای و نبودن هیچ نشانه‌ی شناسایی در آرا، هیچ راهی برای فهمیدن رای یک فرد خاص - حتی اگر برگه‌های رای به دست رقیب بیفتد - وجود ندارد. 
\\
احزار هویت و شمارش رای‌ها به عهده‌ی برگزارکننده‌ی انتخابات است و تنها از طریق یک شخص ثالث برای باز‌شماری آرا می‌توان از اجرای درست آن‌ها اطمینان حاصل کرد.
\\
با توجه به هزینه‌ی زیاد شمارش در انتخابات‌های بزرگ راهی برای اعلام لحظه‌ا‌ی نتایج با هزینه‌ی معقول وجود ندارد.
\\
همانطور که می‌بینیم در روش‌های فعلی انتخابات بسیاری از شرایط مورد نیاز یک انتخابات خوب  با هزینه‌ی نسبتا زیاد فراهم می‌شود. از دیگر مشکلات انتخابات به این روش می‌توان به نیازمندی به یک برگزارکننده‌ی مورد اعتماد اشاره کرد. باید به برگزارکننده اعتماد شود تا: 
\begin{enumerate}
	\item 
	امنیت حوزه‌ی انتخابات را تامین کند.
	\item 
	افراد را به درستی احراز هویت کند.
	\item 
	همه‌ی رای‌ها را بشمارد.
	\item 
	تغییری در رای‌ها ندهد.
	
\end{enumerate}
 

\section{مشکلات و چالش‌های رای‌گیری الکترونیک}
\par
دو مسئله‌ی اساسی در یک سیستم رای‌گیری امنیت و حریم خصوصی است. مخالفین رای‌گیری الکترونیک از کم هزینه بودن تقلب و تغییر رای‌های ثبت شده در انتخابات الکترونیکی می‌گویند و رد کاغذی در یک انتخابات را یک فاکتور مهم برای امنیت آن می‌دانند. هزینه تغییر میلیون‌ها رای در یک سیستم کامپیوتری بسیار پایین‌تر از تولید چند میلیون رای کاغذی تقلبی برای تغییر نتیجه‌ی یک انتخابات است.
\\
بزرگترین مسئله در به‌کارگیری رای‌گیری الکترونیک مسئله‌ی اعتماد به یک سیستم‌ کامپیوتری است. از نظر بسیاری از رای‌دهندگان رای دادن با کامپیوتر شخصی می‌تواند ریسک تغییر رای تا رسیدن آن به سرور‌های رای‌گیری ایجاد کند. از طرف دیگر عدم امکان بررسی و تایید انسانی عملیات کامپیوتر، حس امنیت کمتری القا می‌کند.
\par
مسئله‌ی دیگر پرهزینه بودن ساخت زیرساخت‌های رای‌گیری الکترونیک و خطر پیدایش مشکلات امنیتی در هر سیستم کامپیوتری - چه از نظر نرم‌افزار و چه سخت‌افزار - است. این مشکل باعث شده تعدادی از کشور‌ها از جمله هلند، ایرلند و آلمان فرایند ایجاد زیرساخت لازم را شروع کرده و در ادامه این فرایند را ملقی کنند. دلیل اصلی اعلام شده برای این مسائل‌ قابل اتکا نبودن سیستم‌های رای‌گیری الکترونیکی اعلام شده است. 
\\
برای مثال یک تحقیق معروف از دانشگاه‌ \lr{NYU} در سال ۲۰۱۵ 
\LTRfootnote{https://www.brennancenter.org/publication/americas-voting-machines-risk}
توضیح داد که ماشین‌های رای‌گیری الکترونیکی که در ۴۳ ایالت آمریکا استفاده می‌شوند در سال ۲۰۱۶ به دهمین سال استفاده شدن می‌رسند و به دلیل نداشتن بودجه‌ی کافی برای تعمیرات و بروزرسانی، در معرض خطر کرش
\LTRfootnote{crash}
 کردن هستند که می‌تواند باعث کندی فرایند و حتی گاها از دست رفتن را‌ی‌های مردم شود. علاوه بر این، قدیمی بودن دستگاه‌ها می‌تواند ریسک‌های امنیتی ایحاد کند. 

\par
یک مشکل دیگر در پیاده‌سازی‌های بسیاری از رای‌گیری الکترونیک، نیاز به اینترنت و توانایی استفاده از کامپیوتر است. این مسئله می‌تواند دسترسی بسیاری از افرادی واجد شرایط را - به دلیل نقص جسمی و یا عدم توانایی کار با کامپیوتر - محدود کند. در سیستم‌های فعلی که مبتنی بر حوزه‌های رای‌گیری هستند می‌توانند با کمک انسانی در خود حوزه تا حدی این مشکلات را رفع کنند. 
\par
مشکلات مطرح‌ شده موانع بزرگی برای فراگیری سیستم‌های رای‌گیری کاملا الکترونیکی برای انتخابات‌های مهم و بزرگ هستند که یک سیستم رای‌گیری مناسب باید آن‌ها را تا جای ممکن رفع کند. 

\section{انگیزه و هدف}
هدف این تحقیق، طراجی یک سیستم رای‌گیری الکترونیک است که شرایط رای‌گیری ایده‌آل را تا جای ممکن بدون نیاز به اعتماد به شخص ثالث ایفا کند. با فراگیری تکنولوژی بلاک‌چین برای ایچاد سیستم‌های توزیع شده بدون نیاز به اعتماد (برای مثال بیت‌کوین به عنوان یک‌ ارز دیجیتال بدون نیاز به اعتماد)، پلتفرم‌هایی برای رای‌گیری الکترونیک ایجاد شدند که امنیت شمارش آرا را با عمومی ساختن فرایند رای‌گیری تامین می‌کردند. 
\\
با وجودی که راه‌حل ارائه شده‌ی این سیستم‌ها مسئله‌ی اطمینان از شمارش رای‌ها را حل می‌کرد، مسئله‌ی حریم شخصی در این روش‌ها حل نشده است و انتخابات‌های برگزار شده با این سیستم‌ها امنتیت کمتری در قبال ناشناس ماندن رای‌ها ارائه می‌کنند. 
\\
برای مثال حالتی را فرض کنید که یک رای‌دهنده تهدید می‌شود که باید به یک کاندیدای خاص رای بدهد، در سیستم‌های سنتی رای‌گیری به دلیل بی‌نام بودن برگه‌های رای بعد از اتمام فرایند رای‌گیری راهی برای اطمینان حاصل کردن از نتیجه‌ی رای فرد نیست. از طرفی به دلیل امنیت حوزه‌های رای‌گیری راهی برای اطمینان از نتیجه‌ی رای یک نفر در حین فرایند رای‌گیری هم نیست. پس راهی برای محبور کردن یک نفر که به یک کاندیدای خاص رای بدهد وجود ندارد. اما در سیستم‌های مبتنی بر بلاک‌چین هر رای داده شده به امضای الکترونیکی فرد امضا شده است و این موضوع می‌تواند با عمومی شدن بلاک‌چین بعد از رای‌گیری باعث لو رفتن نتیجه‌ی رای آن فرد شود.
\\
این مشکلات مانع بزرگی برای استفاده‌ی فراگیر این سیستم‌ها خواهد بود. هدف ما در این تحقیق ارائه امنیت و هزینه‌ی کم ناشی از استفاده از این روش‌های رای‌گیری، بدون ایجاد ریسک‌های جدید در حریم خصوصی رای‌دهندگان است. 
\par
نتیجه‌ی این تحقیق یک سیستم‌ رای‌گیری الکترونیک است که قیاس با سیستم‌های سنتی انتخابات هزینه‌ها را کاهش خواهد داد. در عین حال کمترین تغییر برای رای‌دهندگان خواهد داشت که باعث افزایش دسترس‌پذیری این سیستم خواهد شد. همچنین تمامی آرا رای‌دهندگان در قبال یک مهاجم خارجی و حتی خود برگزار کننده‌ی انتخابات ناشناس خواهند ماند. 
\\
از طرفی این سیستم یک رد الکترونیک غیرقابل انکار از تمام ارا، در قبال یک بلاک‌چین، ارائه ‌خواهد کرد که توانایی اثبات درستی شمارش را برای شخص ثالث بدون ایجاد خطری برای ناشناسی رای‌ها خواهد داد. 
\\
تمامی این قابلیت‌ها بدون نیاز اعتماد به برگزارکننده‌ی انتخابات خواهد بود و هرگونه تخطی از پروتکل ارائه شده توسط حوزه‌های رای‌گیری قابل ردیابی از طریق اطلاعات ثبت شده در بلاک‌چین خواهد بود.








\cchapter{تعریف مفاهیم}
در این بخش به معرفی بعضی مفاهیم پایه برای این تحقیق می‌پردازیم. در ابتدا با مفاهیم بلاک‌چین و انواع و کاربرد‌های آن آشنا می‌شویم و در ادامه به بررسی اثبات‌های بی‌دانش می‌پردازیم. این دو تکنولوژی ابزارهای تئوری لازم برای ساخت سیستم رای‌گیری امن خواهند بود.
\section{بلاک‌چین}
بلاک‌چین ساختمان‌داده‌ایست که به مانند لینک‌‌لیست
\LTRfootnote{Linked list}
از بلوک‌‌های متوالی تشکیل شده ولی در بلاک‌چین هر بلوک هش
\LTRfootnote{Hash}
عنصر قبلی خود را نیز نگه‌می‌دارد. هدف از این کار ساخت یک ساختار داده‌ی صرفا افزایشی 
\LTRfootnote{Append only}
است که در آن‌ بلوک‌های قبلی تغییرناپذیرند. تغییر هر بلوک باعث تغییر بلوک بعدی خواهد شد و این موضوع تشخیص تغییر در بلوک‌های پیشین را بسیار ساده می‌کند.
\section{درخت مرکل}


برای پیاده‌سازی یک بلاک‌چین معمولا از درخت مرکل
 \LTRfootnote{Merkle tree}
 استفاده‌ می‌شود. درخت مرکل یا درخت هش، نوعی درخت دودویی 
 \LTRfootnote{Binary tree}
 است که در آن هر راس هش فرزندان خود را نگه‌داشته و برگ‌ها هش داده‌ی ذخیره‌شده در خودشان را نگه می‌دارند. این روش نگه‌داری اطلاعات باعث می‌شود که درچه‌ی زمانی بررسی وجود یک بلوک داده در بلاک‌چین از 
 \lr{N}
 به 
$ \log N$
 کاهش یابد. به دلیل این نوع ساختار یک درخت مرکل، هر تغییری در درخت باعت تغییر هش در ریشه‌ی آن خواهد شد و به دلیل رندم بودن خروچی یک هش خوب، هش ریشه‌ی درخت مرکل هیچ ویژگی قابل پیشبینیی ندارد.
 
 \begin{figure}[th!]
 	\centering
 	\includegraphics[width=.7\linewidth]{Hash_Tree.png}
 	\caption {یک درخت مرکل}
 	\label{fig:merkle}
 \end{figure}
 
 \subsection{انواع بلاک‌چین}
 در این تحقیق بلاک‌چین‌ها را از دو نظر دسته‌‌بندی می‌کنیم. بلاک‌چین‌ها می‌توانند عمومی یا خصوصی باشند، در بلاک‌چین‌های عمومی اضافه‌کردن بلوک به بلاک‌چین دسترسی خاصی نمی‌خواهد و هر کسی می‌تواند در آن‌ها بنویسد ولی در بلاک‌چین‌های خصوصی اضافه کردن بلوک صرفا توسط افراد خاص ممکن است. 
 \\
 روش دیگر تقسیم‌بندی ما باز یا بسته بودن بلاک‌چین است که این دسته‌بندی در مورد دسترسی خواندن اطلاعات از بلاک‌چین است. در بلاک‌چین‌های بسته خواندن اطلاعات توسط عموم آزاد نیست و در بلاک‌چین‌های خصوصی تمام اطلاعات بلاک‌چین برای خواندن، در دسترس عموم است. 
\par
با توجه به کاربرد بلاک‌چین مورد نظر هر بلاک‌چین می‌تواند در هر کدام از این دسته‌بندی‌ها قرار بگیرد، جدول  \ref{tab:tch} یک کاربرد ممکن برای هر کدام از این دسته‌بندی‌ها را نشان می‌دهد.

\begin{table}[h]
	\begin{center}
		%		\def\arraystretch{2}
		\caption{نمونه‌ای جدول}
		\begin{tabular}{|c|c|c|}
			\hline
			& باز & بسته \\
			\hline
			عمومی & ارز‌های دیجیتال & بعضی رای‌گیری‌ها \\
			\hline
			خصوصی & سامانه‌ی مدیریت اطلاعات مالیات & اطلاعات خصوصی یک شرکت \\
			\hline

		\end{tabular}
		\label{tab:tch}
	\end{center}
\end{table}


 
 
\section{اثبات‌های بی‌دانش}
اثبات‌ بی‌دانش 
\LTRfootnote{Zero knowledge proofs}
روشی است که یک «اثبات‌کننده» می‌تواند یه یک «بررسی‌کننده» نشان دهد که او یک راز - مثلا خروجی یک عملیات کامپیوتری - را می‌داند، بدون این که به بررسی‌کننده هیچ اطلاعات اضافه‌ای، مانند خروجی عملیات، بدهد. به عبارت دیگر اثبات‌های بی‌دانش، صرفا داشتن اطلاعات را اثبات می‌کنند و خود اطلاعات را محفوظ نگه‌ می‌دارند.
\\
یک اثبات بی‌دانش باید ۳ شرط زیر را داشته باشد:
\begin{itemize}
	\item
	کامل‌بودن: اگر گزاره‌ی مورد اثبات صحیح باشد، بررسی‌کننده‌ای که پروتکل را رعایت کند، باید از درستی گزاره مطمئن شود.
	\item 
	درستی: اگر گزاره مورد اثبات غلط باشد، هیچ اثبات‌کننده‌ای نتواند اثباتی ارائه کند که گزاره درست است. 
	\item 
	بی‌دانش: اگر اثبات درست باشد، بررسی کننده هیچ اطلاعاتی فراتر از این که گزاره صحیح است دریافت نکند.
\end{itemize}
اثبات‌های بی‌دانش، اثبات‌های احتمالاتی هستند و در واقع احتمال کمی وجود دارد که بتوان یک اثبات نادرست ارائه کرد. به بیان دیگر شرط درستی این است که احتمال تولید یک اثبات نادرست بسیار کم باشد. 
\subsection{مثال شهودی}
سناریویی را در نظر می‌گیریم که یک توپ سبز و یک توپ قرمز روی یک میز قرار دارد و آلیس می‌خواهد به باب که کوررنگ سبز و قرمز است ثابت کند که که این دو توپ با هم تفاوت دارند. برای اثبات آلیس چشمش را می‌بندد و باب یا دو توپ را جابجا می‌کند و یا جابجا نمی‌کند. در ادامه آلیس می‌گوید که آیا جای توپ‌ها با هم عوض شده‌اند یا نه. با یک پاسخ درست باب می‌فهمد که آلیس با احتمال ۵۰٪ درست می‌گوید. این فرایند را تا جایی که باب به احتمال دلخواهش برسد ادامه می‌دهند.
\par
یک نکته‌ی مهم در مثال بالا این است که حتی اگر باب این فرایند را ضبط کرده باشد، نمی‌تواند به کس دیگری اثبات کند که آلیس تفاوت این دو توپ را می‌داند چون که راهی برای اثبات این که سوال و جواب از قبل هماهنگ نشده بوده است ندارد. 
\\
این یکی از نیازمندی‌های بی‌دانش بودن اثبات است. اگر در فرایند برای تصمیم‌گیری در تعویض توپ‌ها باب از شیر یا خط کردن یک سکه استفاده می‌کرد، دیگر این اثبات بی‌دانش نبود، چرا که باب می‌توانست با ضبط کردن این فرایند به یک شخص ثالث اثبات کند که آلیس تفاوت این دو توپ را می‌داند. 
\\
برای داشتن شرط بالا یک اثبات بی‌دانش همواره تعامل از سمت بررسی‌کننده نیاز دارد. اما با ریلکس کردن این شرط و استفاده از یک ورودی غیرقابل پیشبینی برای تولید سوال‌های یک اثبات بی‌دانش - مثلا هش ریشه‌ی یک درخت مرکل - می‌توان اثبات‌های بی‌دانش بدون نیاز به تعامل بررسی‌کننده ساخت. 


\subsection{اثبات‌های بی‌دانش بدون تعامل} 
منظور از اثبات بدون تعامل، اثباتی‌ است که در آن نیازی به فرستادن پیامی از سمت بررسی‌کننده به اثبات‌کننده نباشد. با این روش‌ها اثبات‌کننده می‌تواند اثبات را مستقل از بررسی‌کننده بسازد و ارسال کند، در ادامه‌ی این تحقیق اثبات‌های بی‌دانش و بی‌تعامل را 
\textbf{شاهد}
می‌نامیم. در ادامه دو روش تولید یک شاهد بی‌دانش را بررسی می‌کنیم. این روش‌ها می‌توانند برای خروجی هر محاسبات کامپیوتری شاهد ایجاد کنند. 

\subsubsection{ZK-SNARK}
این روش مخفف
\lr{Zero-Knowledge Succinct Non-Interactive Argument of Knowledge}
است. شاهد‌های این روش علاوه بر بی‌دانش بودن ویژگی‌های زیر را دارند:
\begin{itemize}
	\item 
	مختصر
	\LTRfootnote{Succinct}
	: تولید و بررسی شاهد از انجام خود محاسباتی که اثبات می‌شود کوتاه‌تر (معمولا از مرتبه‌ی زمانی $ (\log N) ^ 2$) است. 
	\item
	بی‌تعامل
	\LTRfootnote{Non-Interactive}
	: نیازی به پیامی از بررسی‌کننده برای ایجاد شاهد نیست. 
	\item
	ادعای دانش
	\LTRfootnote{Argument of Knowledge}
	: اثبات ارائه شده در این روش درست 
	\LTRfootnote{Sound}
	است و نمی‌شود بدون داشتن اطلاعات آن را در زمان محدود ساخت.
	
\end{itemize}
 
\begin{figure}[bh]
	\centering
	\includegraphics[width=.5\linewidth]{arithmetic-circuit.png}
	\caption {یک نمونه مدار محاسباتی}
	\label{fig:arithmetic}
\end{figure}

برای ساختن یک شاهد به این روش ابتدا محاسبات لازم را به یک مدار محاسباتی ریاضی تبدیل می‌کنیم به طوری که اثبات را به عنوان تعدادی شرط روی این مدار نشان دهیم، سپس به کمک یک
\lr{elliptic curve}
مقدار مدار را در چند نقطه‌ی تصادفی به عنوان اثبات ارائه می‌کنیم، با صادق بودن شرط‌ها در این نقاط شاهد را بررسی می‌کنیم. 
\\
برای انتخاب یکسان این نقاط تصادفی بین اثبات‌کننده و بررسی‌کننده نیاز به تعدادی نقطه‌ی توافق شده روی \lr{elliptic curve} داریم که باید در قبل از تولید اثبات انتخاب شده باشند. در این فاز آماده‌سازی تعدادی عدد تصادفی برای انتخاب این نقاط تولید می‌شوند که بعد از تولید نقاط باید بلافاصله پاک شوند. کسی که این اعداد (در واقع نقطه‌ی شروع روی منحنی) را داشته باشد می‌تواند شاهد‌های تقلبی ایجاد کند. برای تولید شاهد واقعی نیازی به دانستن این نقاط نیست و بنابراین بعد از فاز آماده‌سازی این اعداد باید پاک شوند. 
\subsubsection{ZK-STARK}
این روش مخفف
\lr{Zero-Knowledge Scalable Transparent ARguments of Knowledge}
است. مهم‌ترین وجه تمایز این روش در مقایسه با
\lr{ZK-SNARK}
 «شفافیت»
 \LTRfootnote{Transparency}
است، به این معنی که نیازی به فاز آماده‌سازی ندارد. عدم نیاز به آماده‌سازی و نداشتن زباله‌ی سمی (اطلاعاتی که باید پاک شوند تا امنیت سیستم تامین شود) این روش را برای کاربرد‌های حساس مناسب‌تر می‌کند اما در ازای این امنیت، حجم شاهد‌ها از چند صد بایت به چند صد هزار بایت تغییر می‌کند.
\par
ار مزیت‌های دیگر این روش استفاده نکردن از 
\lr{Elliptic curve}ها
است. نیاز‌های کم این روش باعث می‌شود که حتی با کامپیوتر‌های کوانتمی
\LTRfootnote{Quantum computers}
 راهی برای شکستن این اثبات‌ها وجود نداشته باشد.
\\
برای ساختن یک شاهد با این روش، برنامه‌ی مورد نظر را تبدیل یه یک چندچمله‌ای درجه بالا می‌کنند، سپس از مقدایر این چندجمله‌ای یک درخت مرکل ساخته می‌شود که مقدایر مختلف خروچی را نشان می‌دهد. سپس بررسی‌کننده چند شاخه از این درخت را به طور تصادفی انتخاب و بررسی می‌کند. برای غیرتعاملی کردن این اثبات می‌توان از هش ریشه‌ی درخت مرکل به عنوان ورودی یه تابع شبه‌تصادفی
\LTRfootnote{Pseudo random}
استفاده می‌شود که مشخص می‌کند خروجی کدام شاخه‌ها باید در شاهد بیاید. 





\cchapter{کارهای پیشین}
در این بخش ابتدا به بررسی تحقیقاتی می‌پردازیم که به مسئله‌ی حدف اعتماد از سیستم‌های مبتنی بر اعتماد پرداخته‌اند، در ادامه به کارهای مربوط به رای‌گیری الکترونیک و در نهایت به اثبات‌های بی‌دانش می‌پردازیم.

\section{اعتماد}
مسئله‌ی حذف نیاز به یک شخص معتمد را از طریق عمومی ساختن کل اطلاعات مورد نیاز می‌توان حل کرد. اگر تمامی اطلاعات درست باشد هر کسی می‌تواند برای خود درستی تراکنش‌ها را بررسی کند. مسئله‌ای که باقی می‌ماند زمانیست که بین چند شخص اختلاف پیش می‌آید که نسخه‌ی درست اطلاعات کدام است. مثلا زمانی که چند نسخه‌ی صحیح از نظر فرمت وجود دارند اما نتایج مختلفی را می‌رسانند. 
\subsection{توافق}
توصیف رسمی این مسئله، مسئله‌ی ژنرال‌های بیزنتین 
\LTRfootnote{The Byzantine genarals problem}
\cite{byzantine}
است. در این مسئله چند ژنرال که می‌توانند یک به یک با هم صحبت کنند، در تلاشند تا به توافق برسند که آیا باید حمله کنند یا نکنند، تعدادی از ژنرال‌ها خائن هستند و در تلاشند که نتیجه‌ی توافق ژنرال‌ها را تغییر دهند. ژنرال‌های خائن می‌توانند با جواب ندادن یا جواب غلط دادن تلاش کنند که نتیجه‌ی توافق را تغییر دهند. در ساده‌ترین حالت و بدون استفاده از امضا‌های دیجیتال ثابت می‌شود که برای $ 3k + 1 $ ژنرال، با رای‌گیری می‌توان تا $ k $ خائن را تحمل کرد. 
\par
راه‌حل‌های متعددی برای توافق 
\LTRfootnote{اثبات‌ کار}
در بستر بلاک‌چین داده شده که در ادامه به تعدادی از آن‌های می‌پردازیم.
\subsubsection{توافق}
روشی که 
\lr{S.Nakomoto}
\cite{bitcoin}
برای رفع این مسئله در بیت‌کوین استفاده کرده است، اثبات کار 
\LTRfootnote{Proof of work}
نام دارد. این روش که بر پایه‌ی روش استفاده شده در 
\lr{hashcash}
\cite{hashcash}
است. در این روش برای اضافه شدن هر بلوک به بلاک‌چین باید یک مسئله‌ی سخت (که نیاز به توان پردازشی بالا دارد) حل شود ولی بررسی درستی جواب ساده است. این روش روش بسیار فراگیری در ارز‌های دیجیتال است. از مشکلات این روش می‌توان به توان مصرفی بالا و کندی نسبی آن اشاره کرد. برای مثال حداکثر توان تئوری بیت‌کوین، ۷ تراکنش بر ثانیه است. 
\subsubsection{اثبات سهم}
در روش اثبات سهم
\LTRfootnote{proof of stake}
\cite{PoS}
برای ساخت بلوک‌های جدید باید یک فاکتور مقدار سکه‌های در اختیار ماینتر و سن آن‌هاست. به این صورت که می‌تواند در ازای سن‌ سکه‌های در اختیارش (با زدن یه تراکنش به خود) هش ساده‌تری برای بلوک بعدی اعمال کند. مزیت اصلی این روش توان مصرفی پایین‌تر آن به نسبت اثبات کار است. 
\\
معمولا در بلاک‌چین‌ها در بلوک‌های ابتدایی ار روش اثبات کار استفاده می‌شود و بعد از مدتی برای کاهش هزینه‌های اضافه کردن بلوک چدید و مقایس‌پذیری می‌توان از این روش یا ترکیب این روش‌ها استفاده کرد.

\subsubsection{Ripple Consensus Protocol}
در این روش
\cite{ripple} \cite{ripple2}
 تعدادی شخص مورد اعتماد وجود دارند که برای اضافه‌شدن بلوک به بلاک‌چین باید درصدی از آن‌ها درستی تراکنش را تایید کنند. این اشخاص در دسته‌های مختلف قرار می‌گیرند و برای تایید باید یک زیردسته‌ی کامل تراکنش‌ها را تایید کنند.
 \\
 با وجود سرعت نسبتا بالای این روش - تا ۱۰۰۰ تراکنش در ثانیه - منتقدین آن از نیاز به اشخاص مورد اعتماد می‌گویند. این روش تا $ n /5 $ خطا در نود‌های مورد اعتماد را می‌تواند تحمل کند.


\subsubsection{Stellar Consensus Protocol}
 روش 
\lr{SCP}
\cite{scp}
 مبتنی بر ایجاد افراد مورد اعتماد، به صورت طبیعی و خودجوش، در شبکه است. در این روش با افزایش تراکنش‌های درست توسط هر شخصی، آن شخص به عنوان فرد مورد اعتماد شناخته می‌شود و هر تراکنش را باید تعداد افراد مورد اعتماد تایید کنند. این افراد توسط پرداخت‌کننده‌ی تراکنش انتخاب می‌شوند اما دسته‌بندی آن‌ها در شبکه‌ به گونه‌ای است که خطا در تایید تراکنش باعث حذف شدن فرد از لیست افراد مورد اعتماد شود. تفاوت اصلی این روش با 
 \lr{Ripple}
 در توانایی انتخاب تاییدکنندگان تراکنش است و فرض‌های اعتماد کمتر این روش باعث می‌شود که تا $ n /3 $ خطا در نود‌های مورد اطمینان را بتواند تحمل کند.



\subsection{کاربرد‌های بلاک‌چین}
\subsubsection{ارز دیجیتال}
اولین کاربرد بلاک‌چین، بیت‌کوین بود که با موفقیت آن چندین مجصول دیگر هم تولید شدند. 
در بستر بیت‌کوین اثبات کار به صورت زیر استفاده می‌شود.:
\\
هر بلوک جدید شامل تعدادی تراکنش، برای ثبت در بلاک‌چین است. اما برای پذیرفته شدن این بلوک توسط دیگر دیگران، باید در این بلاک یک رشته‌ي بی‌معنی 
\LTRfootnote{Nounce}
قرار گیرد به صورتی که هش بلاک از عددی که توسط پروتکل بیت‌کوین انتخاب می‌شود کمتر باشد. این شرط در طول زمان به صورت خودکار به روزرسانی می‌شود به طوری که در هر لحظه به صورت میانگین اضافه کردن یک بلوک ۱۰ دقیقه از کل شبکه زمان ببرد. از آنجایی که تنها راه یافتن همچین رشته‌ای بروت‌فورس
\LTRfootnote{Bruteforce}
است، توان محاسباتی بالاتر احتمال یافتن بلوک بعدی را افزایش خواهد داد. 
\par
در پروتکل بیت‌کوین طولانی‌ترین بلاک‌چین - یعنی بلاک‌چینی که بیشترین توان محاسباتی برای آن صرف شده - به عنوان نسخه‌ی درست در نظر گرفته می‌شود. در نتیجه 
فرض می‌کنیم شخص A یک بیت‌کوین را به B منتقل کرده، این تراکنش در بلاک‌چین ثبت شده و در ازای آن کالایی دریافت کرده‌است، حال قصد دارد این تراکنش را از بلاک‌چین بیت‌کوین حذف کند تا بتواند آن را دوباره خرج کند. از آنجایی که نود‌های شبکه‌ی بیت‌کوین اگر ۲ زنجیره از بلوک‌ها دریافت کنند زنجیره‌ی بلند‌تر را قبول خواهند کرد باید قبل از این که کل شبکه یک بلوک  اضافه کند، دو بلوک سالم بسازد.
\\
احتمال موفقیت حمله‌ی A مساوی
$(\frac{A's\ computational\ power}{Bitcoin\ network's\ computational\ power}) ^ 2 $
است. اگر توان محاسباتی A از دیگر قسمت‌های شبکه کمتر باشد، این کسر عددی کوچک‌تر از 
\lr{0.5}
است. اگر این کار به موقع با موفقیت انجام نشود، سه بلوک عقب می‌افتد و توان فرمول بالا تبدیل به سه می‌شود و احتمال موفقیتش کمتر از پیش می‌شود. 
\\
این مسئله مسئله‌ی قمارباز
\LTRfootnote{Gambler's Ruin}
نام دارد که نشان داده می‌شود در آن در طول زمان احتمال موفقت مهاجم به صورت نمایی کاهش پیدا می‌کند.
\par

از پرکاربرد‌ترین‌‌های این محصولات می‌توان به
\lr{LiteCoin}
و 
\lr{z-cash}
\cite{zerocash}
اشاره کرد. مزیت لایت‌کوین نسبت به بیت‌کوین در هزینه‌ی کمتر اضافه کردن بلوک و زمان تراکنش‌های کمتر است و زی‌کش امنیت بیشتری در زمینه‌ی حریم خصوصی و ناشناس ماندن ارائه می‌کند. 
\\ 
از نمونه‌های دیگر می‌توان به 
\lr{Ripple}
اشاره کرد، که برای تراکنش‌های بانک‌ها طراحی شده و همانطوری که بررسی کردیم از اثبات کار برای توافق استفاده نمی‌کند. 

\subsection{سازمان‌های توزیع‌شده‌ی خودکار}
دسته‌ی بعدی کاربرد‌های بلاک‌چین ایجاد سازمان‌های توزیع‌شده‌ی خودکار 
\LTRfootnote{Decentralized autonomous organization (DAO)}
است. این مفهوم بر اساس مفهوم قرارداد هوشمند
\LTRfootnote{Smart contract}
\cite{SmartContract}
 ساخته شده است. این سازمان‌ها برنامه‌هایی هستن که به طور خودکار و بی‌اعتماد در یستر بلاک‌چین ایجاد می‌شوند و می‌توانند کاربرد‌های بسیاری داشته باشند. اتریوم
 \LTRfootnote{Ethereum}
\cite{Ethereum}
که یک ارز دیجیتال است، یک بستر مناسب برای تولید قرارداد‌های هوشمند هم هست که در سال‌های اخیر کاربرد‌های بیشتری پیدا کرده‌اند. با این وجود یک دغدغه‌ای که همچنان در این بستر وجود دارد خطر اشتباه‌های برنامه‌نویسی در این سازمان‌هاست، از آن‌جا که تغییر کد اینگونه سازمان‌ها به دلیل خودکار بودن، همواره قابل تغییر نیست، اشتباهات امنیتی می‌تواند تاثیر مخرب عظیمی داشته باشند. atezi
\cite{surveyAtt}
به بررسی مشکلات امنیتی معمول قراردادهای در بستر اتریوم و تله‌ی معمول این زبان برنامه‌نویسی و روش‌های تصحیح آن‌ها پرداخته است. 
\\
یک‌ مسئله‌ی دیگر که ناشی از ناشناسی ذاتی این بستر‌هاست ایجاد شدن سازمان‌های مچرمانه در آن‌هاست. تحقیقات‌ بسیاری 
\cite{gyges} \cite{smart}
در این زمینه شده، و از مثال‌های قرارداد‌های مخرب ممکن می‌توان به افشای اطلاعات خصوصی و یا حتی دزدین کلید‌های رمز‌نگاری اشاره کرد. به دلیل عدم وجود نظارت مرکزی در این سیستم‌ها، راهی برای جلوگیری از اینگونه قرارداد‌ها وجود ندارد.
\subsection{شناسایی}
از کاربرد‌های دیگر بلا‌ک‌چین ساخت‌ روش‌های شناسایی
\LTRfootnote{Authentication}
است. ویژگی تغییر ناپذیری بلوک‌های قدیمی در یک بلاک‌چین، باعث می‌شود که ساختارداده‌ی ایده‌آلی برای شناسایی باشد. از این نمونه کاربرد‌ها می‌توان به \lr{namecoin} اشاره کرد. هدف این محصول کاهش نیاز به اعتماد در 
\lr{DNS} \LTRfootnote{Domain Name Service}
است. این محصول که روی بلاک‌چین بیت‌کوین ساخته شده است، در متن نوشته‌شده در تراکنش اطلاعات مربوط به نام دامنه‌ها را زخیره می‌کند. 
\par
از تخقیقات دیگر در این زمینه می‌توان به امنیت در داکر
\LTRfootnote{Docker}
\cite{docker}
اشاره کرد. در این تحقیق با استفاده‌ از بلاک‌چین یک بستر توزیع‌شده برای به اشتراک‌گذاری فایل‌های امضا شده ساخته شده است. در این سیستم‌ می‌توان از یک بلاک‌چین خصوصی و یا یه بلاک‌چین عمومی استفاده کرد. 


\section{رای‌گیری الکترونیک}
سیستم‌های رای‌گیری الکترونیک را می‌توان به دو دسته‌ی کلی توزیع‌شده و متمرکز تقسیم کرد. سیستم‌های مرکزی نیازمند یک ارتباط امن از رای‌دهنده تا سرویس مرکزی هستند. همچنین نیازمند اعتماد کامل به همان یک سرویس برای درستی انتخابات است. در سیستم‌های توزیع‌شده تلاش می‌کنند تا این دو مسئله را کمرنگ‌تر کنند.

\subsection{رای‌گیری الکترونیک متمرکز}
در این سیستم‌ها یک سامانه‌ی مرکزی وجود دارد که تمامی آرا در آن زخیره می‌شند. در این سیستم‌ها حوزه‌های رای‌گیری می‌توانند وجود داشته باشند اما حوزه‌ها صرفا وظیفه‌ی احراز هویت و ارائه‌ی درگاه امن برای ثبت رای در سامانه‌ی مرکزی را دارند. حریم خصوصی کاربران در این سیستم‌ها مبتی بر استفاده از کانال‌های ارتباطی ناشناس 
\LTRfootnote{Anonymous communication channel}
بین حوزه‌های رای‌گیری و سامانه‌ی مرکزی است.
\par
 اولین تحقیق در رابطه با استفاده‌ از کانال‌های ارتباطی ناشناس برای رای‌گیری در سال ۱۹۸۵ توسط 
\lr{Chaum}
\cite{Chaum}
بود که در آن برای ساخت‌ کانال‌های ارتباطی ناشناس از امضای کورکورانه 
\LTRfootnote{Blind signature}
\cite{blindsig}
برای ایجاد یک ارز دیجیتل استفاده می‌شد. در امضای کورکورانه، برای حفظ حریم خصوصی فردی که باید اطلاعاتی را تایید کند، رمزشده‌ی اطلاعات را امضا می‌کند، به این صورت از اطلاعات پیام باخبر نمی‌شود. اولین تلاش برای ایجاد یک پروتکل رای‌گیری الکترونیک با امضای کورکورانه در سال ۱۹۹۲ 
\cite{foo92}
بود و در ادامه در سال ۱۹۹۷
\cite{improveblind}
نسخه‌ی کامل‌تری از آن ارائه شد. این روش‌ها مبتنی بر وجود یک شمارنده و یک حوزه هستند، حوزه احراز هویت را انجام می‌دهد و برگه‌ رای‌های ناشناس صادر می‌کند و سپس از طریق یک کانال ارتباطی ناشناس رای‌دهنده رای را به شمارنده می‌دهد. از مشکلات این روش می‌توان به نیاز به اعتماد به حوزه اشاره کرد. حوزه می‌تواند که با ارائه رای‌های اشتباه بدون توانایی پیگیری، رای‌گیری را خراب کند، برای حل این مشکل تحقیقاتی
\cite{multiteller}
 در راستای استفاده از چند حوزه انجام شده است. مشکل بزرگ دیگر این روش‌ها
\cite{anonchan}
  سختی ناشناس نگه‌داشتن کانال‌های ارتباطی ناشناس است.

\subsection{رای‌گیری الکترونیک توزیع‌شده}
رای‌گیری الکترونیک توزیع‌شده را به دو دسته‌ی رای‌گیری‌های بدون بلاک‌چین و با بلاک‌چین عمومی و با بلاک‌چین خصوصی تقسیم می‌کنیم. با توجه به این که در رای‌گیری احرازهویت یک مسئله‌ی مهم است همه‌ي این سیستم‌ها از بلاک‌چین‌های بسته استفاده می‌کنند.
\subsubsection{رای‌گیری بدون بلاک‌چین} 
بعضی سیستم‌های طراحی شده برای رای‌گیری الکترونیک
\cite{secret1}
\cite{secret2}
\cite{secret3}
 از روش‌های تقسیم راز
\LTRfootnote{Secret sharing}
 استفاده می‌کنند. در این روش‌ها فرایند ثبت رای باید به تایید تعدادی از حوزه‌ها برسد که باعث کاهش نیاز به اعتماد می‌شود. در این روش‌ها می‌توان اثبات کرد که برای ردیابی یک رمز حداقل $k$ حوزه باید تبانی کنند. 
 \par
 روش دیگری که برای رای‌گیری توزیع‌شده استفاده شده است، استفاده از بردار‌ بررسی
 \LTRfootnote{Check vector}
 \cite{checkvector}
 است. در این روش‌ها بررسی درستی رای‌ها کاملا توزیع‌شده‌ است اما نیاز به ارتباط دو به دوی تمامی رای‌دهنده‌ها دارد که در یک انتخابات واقعی شدنی نیست. ترکیبی از این روش و تقسیم راز باعث ایجاد پروتکل‌هایی
 \cite{MPO1} \cite{evotinwocrypto}
  شد که با سطح‌بندی حوزه‌ها و تقسیم رای‌ها و مخلوط کردن آن‌ها از حریم خصوصی حمایت می‌کنند. اما این روش‌ها به حوزه‌ها و رای‌دهندگان توانایی بررسی درستی برگه رای‌ها را نمی‌دهد و امکان ایجاد رای‌های اشتباه و جلوگیری از رای‌دادن یک فرد خاص را ایجاد می‌کنند. 


\subsubsection{رای‌گیری با بلاک‌چین عمومی}
با فراگیر شدن تکنولوژی بلاک‌چین
\cite{rosgood}
، محصولاتی در زمینه‌ی رای‌گیری الکترونیک به کمک این تکنولوژی ساخته شدند. تعدادی از این سیستم‌های رای‌گیری در قالب قرارداد‌های هوشمند ساخته‌ شنده‌اند که از آن‌ها می‌توان به وتریم 
\LTRfootnote{Votereum}
\cite{votereum}
و یا کار 
\lr{E.Yavuz}
\cite{yavuz}
در بستر اتریوم اشاره کرد. مزیت اینجور رای‌گیری‌ها هزینه‌ی اولیه کم استفاده از آن‌هاست، اما همچنان ریسک اشتباه برنامه‌نویسی در این سبک کارها بسیار بالاست. همچنین هزینه اجرای قراردادهای هوشمند به تعداد بالا برای یک رای‌گیری هزینه‌ی بالایی خواهد داشت که در طول زمان باعث افزایش هزینه‌ی رای‌گیری خواهد شد. مسئله‌ی دیگر در بستر اتریوم هم وابستگی سیستم رای‌گیری، به پهنای باند نود‌های اتریوم و میزان بار روی شبکه‌ی آن است. این موضوع می‌تواند باعث کند شدن یا حتی در مواردی حذف شدن تعدادی از آرا شود.


\subsubsection{ٰرای‌گیری با بلاک‌چین خصوصی}
از سیستم‌های رای‌گیری با بلاک‌چین خصوصی می‌توان به 
\lr{VoteBook}
\cite{votebook}
توسط شرکت 
\lr{Kaspersky}
که یک شرکت پیشرو در زمینه‌ی امنیت است اشاره کرد. فلسفه‌ی ساخت این سیستم به صورتی است که تلاش می‌کند برای کاربرانی که از سیستم‌هایی رای‌گیری فعلی استفاده می‌کنند کمترین تغییر در رفتار نیاز باشد. 
\\
از مثال‌های دیگر سیستم‌های رای‌گیری مبتنی بر بلاک‌چین می‌توان به استارت‌آپ 
\lr{Follow My Vote}
اشاره کرد. نجوه‌ی کار این سیستم با سیستم 
\lr{VoteBook}
تفاوت اساسی دارد و برای رای‌دادن احتیاج دارد که نرم‌‌افزاری برای رای‌دادن به روی کامپیتور و یا تلفن‌همراه کاربران نصب شود. اینگونه طراحی سیستم،‌ خطرات امنیتی در قالب بدافزار ایجاد می‌کند. همچنین با نبود یک حوزه‌ی رای‌گیری امن راهی برای تامین امنیت رای‌دهندگان و اطمینان حاصل کردن از این که کسی مجبور به رای‌ دادن نشده، نیست.
\\
و در نهایت یکی از موفق‌ترین سیستم‌های رای‌گیری مبتنی بر بلاک‌چین موجود در حال حاضر 
\lr{VoteWatcher}
ساخته شده توسط یک شاخه از شرکت 
\lr{blockchain Technologies Corporation}
است که یک شرکت بزرگ برای ارائه‌ی سرویس‌های مبتنی بر بلاک‌چین است. طبق وب‌سایت این محصول تاکنون بیش از صدهزار رای در بیشتر از ۲۰ رای‌گیری مختلف توسط این سیستم‌ شمارش شده‌است. 
\\
مدل اسفاده‌ی 
\lr{VoteWatcher}
به سیستم‌ 
\lr{VoteBook}
بسیار شبیه است و تفاوت رفتاری زیادی با مدل‌های رای‌گیری الکترونیکی فعلی برای کاربران ندارد. در این محصول طبق نیاز رای‌گیری می‌توان از یک بلاک‌چین عمومی یا خصوصی استفاده کرد.
\\
از موارد دیگر می‌توان به پیاده‌سازی‌های به کمک قرارداد‌های هوشمند ولی با استفاده از بلاک‌چین خصوصی 
\cite{privblock}
که توانایی ردگیری بالاتری از پیاده‌سازی‌های دیگر ارائه می‌کنند، اما به دلیل کندی نسبی، نیاز به تقسیم رای‌گیری به چند رای‌گیری کوچک‌تر دارند.

\section{اثبات‌های بی‌دانش}
اثبات‌های بی‌دانش اولین بار در سال ۱۹۸۵ 
\cite{GHY}
به عنوان روشی ساخت یک روش رمزنگاری متقارن با کلید عمومی استفاده شد. با پیشرفت تکنولوژی اثبات‌های بی‌دانش،‌ روش‌های جامع اثبات بی‌دانش مانند
\lr{ZK-SNARK} 
\cite{zksnark}
و \lr{ZK-STARK} 
\cite{zkstark}
بوجود آمدند. این روش‌ها توانایی اثبات هر محاسباتی را به صورت بی‌دانش دارند و این موضوع باعث استفاده‌ی آن‌ها در کاربردهای بیشتری شد.
\subsection{کاربرد‌ها}
با فراگیری ارز‌های دیجیتال، مسئله‌ی حریم خصوصی در آن‌ها پررنگ‌تر شد. در ابتدا یکی از بزرگ‌ترین کاربردهای بیت‌کوین پرداخت‌های مجرمانه بود که ناشناسی نسبی در این بستر باعث می‌شد برای این سبک‌ پرداخت‌ها ایده‌آل باشند. اما به دلیل عمومی بودن بلاک‌چین و تمامی تراکنش‌ها در آن دنبال کردن رد پرداخت بسیار ساده است.
\par
 برای جلوگیری از این ردگیری در کاربردهای مجرمانه از یک شخص مورد اعتماد برای «مخلوط کردن» سکه‌های افراد استفاده می‌شود. در این روش چندین نفر به یک نفر پرداخت می‌کنند و آن فرد به کلید‌های عمومی که از قبل تعیین شده با سکه‌های جدید پرداخت می‌کند اما در این روش همگی به شخص مخلوط کننده اعتماد می‌کنند که به اندازه پرداخت کند و رد واقعی سکه‌ها را جایی ثبت نکند. این روش برای کاربردهای مجرمانه تا حد بسیار خوبی پاسخگو است اما برای افراد عادی که صرفا دغدغه‌ی حریم خصوصی خود را دارند خطرها و هزینه‌ي این روش معقول نیست، به همین دلیل کارهای مختلفی در زمینه‌ی پرداخت ناشناس انجام شده است. 
 
\subsubsection{پرداخت ناشناس}
از این تحقیقات می‌توان به بستر 
\lr{HAWK}
\cite{hawk}
اشاره کرد. در این تحقیق به کمک یک تعریف کلی از بلاک‌چین به عنوان سیستمی که همواره در دسترس است و هیچ اطلاعات اشتباه نمی‌پذیرد اما حریم خصوصی را حفظ نمی‌کند یک بستر قرارداد هوشمند ساخته شده است که در آن به ازای کد قرارداد هوشمند، یک کد برای حفظ حریم خصوصی به کمک اثبات‌های بی‌دانش ساخته می‌شود. روش کار این سیستم مبتنی بر ایجاد آدرس‌های مقصد یکتا به ازای هر تراکنش است. 
\\
مونرو 
\LTRfootnote{Monero}
که یک ارز دیجیتال است که بر اساس الگوریتم 
\lr{Cryptonote}
\cite{monero}
کار می‌کند، به مانند 
\lr{HAWK}
با یکتا سازی آدرس‌های مقصد و امضای حلقه‌ای 
\LTRfootnote{ًRing Signature}
کار می‌کند. در ادامه این ارز دیجیتال با همین روش مدلی 
\cite{monero2}
برای مخفی کردن پرداخت‌کننده‌ی سکه نیز ارائه کرد.
\par
از کارهای دیگر در این زمینه‌ می‌توان به 
\lr{zerocoin}
\cite{zerocoin}
که روش پرداخت ناشناس بر بستر بیت‌کوین به کمک 
\lr{ZK-SNARK}
ارائه کرد اشاره کرد. روش استفاده شده در آن با بهبود در ارز دیجیتال 
\lr{z-cash}
\cite{zerocash}
استفاده شد. این ارز دیجیتال دو مدل سکه‌ی قابل ردگیری و غیرقابل ردگیری دارد و هر کسی می‌توان طی یک تراکنش سکه‌های خود را به سکه‌های ناشناس تبدیل کند.

. 
\subsection{فاز آماده‌سازی}
با توجه به این که 
\lr{ZK-SNARK}
‌ها، به طور خاص خاص پروتکل پینوکیو
\LTRfootnote{Pinocchio}
فراگیرترین روش برای ایجاد اثبات‌های بی‌دانش است. تحقیقات زیاد در مورد فاز آماده‌سازی و ایجاد پارامترهای عمومی این مدل اثبات شده است. همانطور که قبلا اشاره کردیم ورودی‌های این فاز اگر بعد از این فاز پاک نشوند می‌توانند برای ایجاد اثبات‌های تقلبی استفاده شوند.
\par
از این تحقیق‌ها می‌توان به روش‌هایی
\cite{znsetup}
 که تلاش در کاهش نیاز به اعتماد در این فاز می‌کنند اشاره کرد. این روش‌ها باعث می‌شوند که برای لو رفت اطلاعات خطرناک احتیاج به تبانی تمامی اعضای موجود در فاز آماده‌سازی باشد.
 \par
 از دیگر کارهای در این زمینه می‌توان به تلاش‌هایی برای حذف فاز آماده‌سازی به طور کلی اشاره کرد. این روش‌ با تغییر اساسی در پروتکل و استفاده از چندجمله‌ای‌ها 
 \cite{nosetup}
 مرحله‌ي آماده‌سازی را حذف می‌کند

\cchapter{روش پیشنهادی}
در این بخش به بررسی روش پیشنهادی این تحقیق می‌پردازیم، در ابتدا شرایطی که در آن مسئله را حل کنیم بررسی می‌کنیم و شرط‌های لازم برای سیستم‌ رای‌گیری خود را بررسی می‌کنیم، در ادامه به بررسی کلی روش رای‌گیری عناصر حاضر در آن می‌پردازیم و در نهایت الگوریتم‌ها و پروتکل دقیق را بررسی می‌کنیم.

\section{تعریف عناصر}
در این بخش عناصر حاضر در سیستم رای‌گیری و انتظارات خود از آن‌ها را تعریف می‌کنیم:
\begin{itemize}
	\item
	\textbf{ناظر انتخابات}:
	این سازمان مسئول بررسی درستی انتخابات و احراز هویت شرکت‌کنندگان در انتخابات است. به این سازمان اعتماد می‌شود تا کار احراز هویت را به درستی انجام دهد. همچنین این سازمان بررسی‌کننده‌ی نهایی درست بودن انتخابات است و باید بتواند از درستی انتخابات اطمینان حاصل کند.
	\item
	\textbf{رای‌دهنده}:
	فردی که حق رای به یک کاندیدا را دارد، این فرد می‌تواند از رایش استفاده بکند یا نکند، می‌تواند تلاش کند که چند بار رای‌دهد. باید بتواند از درستی انتخابات اطمینان حاصل کند. ممکن است توسط یک رقیب بدخواه برای رای‌دادن تحت فشار قرار بگیرد.
	\item
	\textbf{حوزه‌ی انتخابات}:
	محلی که در آن رای داده می‌شود، باید بتواند امنیت فیزیکی افراد را تامین کند. ممکن است برای خراب کردن انتخابات یا نقض حریم خصوصی کاربران تلاش کند. 
\end{itemize}
\section{شرایط مسئله}
شروط لازم برای سیستم‌ رای‌گیری ارائه شده عبارتند از:  
\begin{enumerate}
	\item 
	هر فرد واجد شرایط دقیقا یک بار بتواند رای دهد.
	\item 
	هیچ کسی نتواند به جای فرد دیگری رای دهد.
	\item 
	هیچ فردی مجبور به رای دادن نشود.
	\item 
	هیچ فردی مجبور به رای دادن به کاندیدای خاصی نشود.
	\item 
	در صورت نقض حریم خصوصی و یا شمارده نشدن بعضی رای‌ها ناظر انتخابات بتواند حوزه‌ي متخلف را شناسایی کند. حوزه‌ها به دلیل در اختیار داشتن سامانه‌های کامپیوتری رای‌گیری همواره می‌توانند با روش‌های 
	\lr{phishing}
	و یا استفاده از بد‌افزار‌ها حریم خصوصی کاربر را زیر سوال ببرند یا رای او را ثبت نکنند. به همین دلیل قابل پیگیری بودن تخلفات حوزه‌ها یکی از مهم‌ترین شرایط یک انتخابات درست است.
	\item 
	هر رای‌دهنده بتواند به کمک ابزارهای رمزنگاری اطمینان حاصل کند که رای او شمرده شده است. این شرط به این معنی است که هر کاربری که دانش کافی داشته باشد باید بتواند از درستی انتخابات - بدون نیاز به اعتماد به حوزه یا حتی ناظر انتخابات - اطمینان حاصل کند. 
	\item
	رای‌دهنده نیازی به دانش یا توانایی خاصی برای رای‌دادن نداشته باشد. این نیازمندی برای دسترس‌پذیر نگه‌داشتن انتخابات لازم است.
	\item
	سیستم رای‌گیری نسبت به روش‌های فعلی رای‌گیری از دید کاربر تفاوت چندانی نداشته باشد. هر تغییر اساسی از نگاه رای‌هنده باعث سختی نسبی انتخابات خواهد شد و هزینه‌ی اولیه استفاده از این سیستم انتخاباتی را به شدت افزایش خواهد داد.
	\item 
	بتوان نتایج انتخابات را در بازه‌های زمانی معین دید. هر سیستم انتخاباتی که نتایج لحظه‌ای نشان دهد همواره در خطر حمله‌های مبتنی بر زمان در رابطه با حریم خصوصی رای‌دهندگان خواهد بود، به همین منظور نتایج انتخابات را می‌توان در بازه‌های زمانی که حریم خصوصی را به خطر نیندازد نشان داد.
	\item 
	بتوان از شمارده شدن تمامی آرا بعد از انتخابات اطمینان حاصل کرد. به دلیل الزام مخفی نگه‌داشتن زمان حدودی ارسال هر رای، نتایج نهایی انتخابات تنها بعد از اتمام رای‌گیری قابل اتکاست.
\end{enumerate}

\section{فرضیات مسئله}
مسئله‌ی شناسایی یک مسئله‌ی مهم در هر انتخابات است، با توجه به این که افراد واجد شرایط بسته به هر انتخابات تغییر می‌کنند در این مسئله فرض می‌کنیم که هر رای‌دهنده یک جفت کلید خصوصی و عمومی دارد که قبل از فرایند انتخابات توسط ناظر انتخابات تایید شده است. 
\par
وظیفه‌ی حفظ امنیت کلید عمومی و خصوصی هر کاربر به عهده‌ی خود کابر خواهد بود چرا نشانگر هویت کاربر در سیستم‌ کلید عمومی او خواهد بود. هر چند که برای رای دادن در حوزه اطلاعات شناسایی کابر با کلید عمومی او تطابق داده خواهد شد.

\section{مثال شهودی}
برای بدست آوردن دید کلی در راه‌حل ابتدا یک مثال شهودی از یک مدل رای‌گیری متمرکز را بررسی می‌کنیم،‌ سپس در ادامه از این روش برای ایجاد سیستم توزیع‌شده و الکترونیکی خود استفاده می‌کنیم.
\par
یک رای‌گیری را فرض می‌کنیم که در آن به ازای هر رای‌دهنده یک کاغذ نام رای‌دهنده و یک برگه‌ی رای وجود دارد. همچنین یک صندوق به ازای هر کاندیدا وجود دارد و همه‌ی این اطلاعات در معرض دید عموم هستند. 
\par
آلبس برای رای‌دادن برگه‌ی رای خود را بر مي‌دارد و همراه یک کاغذ دیگر به یک تخته می‌چسباند. در برگه‌ی دوم آلیس یک شماره‌ی تصادفی $r$ و کاندیدای موردنظر خود، باب، را می‌نویسد و کل آن را با یک کلید تصادفی رمز می‌کند.
\\
در ادامه آلیس پس از مدتی در حالی که یک نقاب به صورت خود زده به تخته مراجعه‌ی می‌کند و شاهدی بر تخته ثبت می‌کند که ثابت می‌کند او یک کلیدرمزی می‌داند که یکی از کاغذ‌های روی تخته را باز می‌کند که نتیجه‌ی آن $r$ و باب است. سپس یکی از رای‌های روی تخته را برمی‌دارد و به صندوق باب می‌اندازد.
\par
 شاهد آلیس در تخته ثبت شده و دیگر نمی‌توان اثباتی ارائه کرد که رای آلیس را دو بار بشمارد، چرا که عدد $r$ آن تکراری خواهد بود.
\\
از طرفی از آن‌جا که آلیس لزوما برگه‌ی رای خود را برنداشته و نقاب به صورت داشته، راهی برای پیدا کردن نتیجه‌ی رای آلیس نخواهد بود. 
\\ 
چون حوزه‌ی انتخابات محل امنی برای رای‌ دادن است راهی برای مجبور کردن آلیس به رای دادن به کاندیدای خاصی نخواهد بود و با اضافه کردن صندوق رای ممتنع می‌توانیم اطمینان حاصل کنیم که کسی آلیس را مجبور به رای‌دادن نیز نکرده است. 
\section{شمای کلی}
به طور کلی سیستم از تعدادی حوزه‌ تشکیل می‌شود که روی یک بلاک‌چین توافق می‌کنند. بلوک ابتدایی این بلاک‌چین به ازای هر فرد واجد شرایط یک کلید عمومی و یک رای‌دارد. همچنین یک آدرس خروجی به ازای هر کاندیدا وجود دارد که تعداد رای‌هایی که به آن آدرس فرستاده شده باشند رای‌های آن‌ کاندیداست. 
\subsection{تراکنش‌ها}
در این بلاک‌چین دو مدل تراکنش وجود دارد، مدل اول را تراکنش ثبت می‌نامیم و مدل دوم را تراکنش شمارش.
\par
در تراکنش ثبت یک رای از یک کلید عمومی به دسته‌ی رای‌های منتظر شمارش منتقل می‌شود. هر تراکنش ثبت شامل یک رای و یک رشته 
\LTRfootnote{string}
رمز شده است که حاوی یک عدد تصادفی $s$ و حساب مقصد $d$ است که با کلید تصادفی $k$ رمزشده است. این عبارت رمز‌شده رای $CM$ می‌نامیم. سپس $CM$ (نه خود $s$) در بلاک چین به همراه رای ثبت می‌شود.
\\
\begin{equation}
CM = enc_{k} (r, d)
\label{eq:enc}
\end{equation}
\par
مدل دیگر تراکنش ممکن تراکنش شمارش است که در آن شاهدی برای شمارش رای ارائه می‌شود. برای این تراکنش رای‌دهنده اثبات بی‌دانشی برای دو موضوع ارائه می‌کند: یک $C$ می‌شناسد به طوری که  $C \in C_1, C_2, ... ,C_n$ و یک رشته‌ای $r$ می‌داند که $C$ را به $r$ و $d$ باز می‌کند. در نتیجه‌ی این اثبات یکی از رای‌های ثبت‌شده به $d$ منتقل می‌شود.
\par
لازم به ذکر است که از آنجایی که در تراکنش شمارش $C$ و $r$ نشان داده نشده‌اند، راهی برای فهمیدن این که رای متعلق به چه کسی است وجود ندارد.

\par
هر حوزه یک بافر دارید برای نگه‌داری تراکنش‌های مربوط به آرا قبل از ثبت در بلاک‌چین که آن را تبدیل به یک بلوک می‌کند. برای اضافه شدن هر تراکنش روی بلاک‌چین باید حداقل نصف به علاوه‌ی یکی از حوزه‌ها روی آن توافق کنند. هر حوزه برای توافق بررسی می‌کند که که تراکش‌های مربوط به رای دادن و شاهد‌های ثبت شده در بلوک جدید درست باشند و هش بلوک قبلی نیز در بلوک صحیح باشد. 


